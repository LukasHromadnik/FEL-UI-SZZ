% Options for packages loaded elsewhere
\PassOptionsToPackage{unicode}{hyperref}
\PassOptionsToPackage{hyphens}{url}
%
\documentclass[
  czech,
  a4paperpaper,
]{article}
\usepackage{lmodern}
\usepackage{setspace}
\usepackage{amssymb,amsmath}
\usepackage{ifxetex,ifluatex}
\ifnum 0\ifxetex 1\fi\ifluatex 1\fi=0 % if pdftex
  \usepackage[T1]{fontenc}
  \usepackage[utf8]{inputenc}
  \usepackage{textcomp} % provide euro and other symbols
\else % if luatex or xetex
  \usepackage{unicode-math}
  \defaultfontfeatures{Scale=MatchLowercase}
  \defaultfontfeatures[\rmfamily]{Ligatures=TeX,Scale=1}
\fi
% Use upquote if available, for straight quotes in verbatim environments
\IfFileExists{upquote.sty}{\usepackage{upquote}}{}
\IfFileExists{microtype.sty}{% use microtype if available
  \usepackage[]{microtype}
  \UseMicrotypeSet[protrusion]{basicmath} % disable protrusion for tt fonts
}{}
\makeatletter
\@ifundefined{KOMAClassName}{% if non-KOMA class
  \IfFileExists{parskip.sty}{%
    \usepackage{parskip}
  }{% else
    \setlength{\parindent}{0pt}
    \setlength{\parskip}{6pt plus 2pt minus 1pt}}
}{% if KOMA class
  \KOMAoptions{parskip=half}}
\makeatother
\usepackage{xcolor}
\IfFileExists{xurl.sty}{\usepackage{xurl}}{} % add URL line breaks if available
\IfFileExists{bookmark.sty}{\usepackage{bookmark}}{\usepackage{hyperref}}
\hypersetup{
  pdflang={cs-CZ},
  hidelinks,
  pdfcreator={LaTeX via pandoc}}
\urlstyle{same} % disable monospaced font for URLs
\usepackage{longtable,booktabs}
% Correct order of tables after \paragraph or \subparagraph
\usepackage{etoolbox}
\makeatletter
\patchcmd\longtable{\par}{\if@noskipsec\mbox{}\fi\par}{}{}
\makeatother
% Allow footnotes in longtable head/foot
\IfFileExists{footnotehyper.sty}{\usepackage{footnotehyper}}{\usepackage{footnote}}
\makesavenoteenv{longtable}
\setlength{\emergencystretch}{3em} % prevent overfull lines
\providecommand{\tightlist}{%
  \setlength{\itemsep}{0pt}\setlength{\parskip}{0pt}}
\setcounter{secnumdepth}{5}
\usepackage[a4paper,left=2.5cm,right=2.5cm]{geometry}
\ifxetex
  % Load polyglossia as late as possible: uses bidi with RTL langages (e.g. Hebrew, Arabic)
  \usepackage{polyglossia}
  \setmainlanguage[]{czech}
\else
  \usepackage[shorthands=off,main=czech]{babel}
\fi

\author{}
\date{}

\begin{document}

\setstretch{1.2}
\hypertarget{ilp}{%
\section{ILP}\label{ilp}}

The ILP problem is given by matrix
\(\mathbf{A} \in \mathbb{R}^{m \times n}\) and vectors
\(\mathbf{b} \in \mathbb{R}^m\) and \(\mathbf{c} \in \mathbb{R}^n\). The
goal is to find a vector \(\mathbf{x} \in \mathbb{Z}^n\) such that
\(\mathbf{A} \cdot x \leq b\) and \(c^T \cdot x\) is the maximum.

Usually the problem is given as

\[\max \left\{ c^T \cdot x : \mathbf{A} \cdot x \leq b, x \in \mathbb{Z}^n \right\}\]

Since the \textbf{ILP solution psace is not a convex set}, we cannot use
convex optimization techniques.

The most successful method to solve the ILP problem are:

\begin{itemize}
\tightlist
\item
  Enumerative methods
\item
  Branch and bound method
\item
  Cutting planes methods
\end{itemize}

\hypertarget{enumerative-methods}{%
\subsection{Enumerative methods}\label{enumerative-methods}}

Based on the idea of \textbf{inspecting all possible solutions}. Due to
the integer nature of the variables, the number of solutions is
countable but their number is huge. So this method is usually suited
only for smaller instances with a small number of variables.

\hypertarget{branch-and-bound-method}{%
\subsection{Branch and bound method}\label{branch-and-bound-method}}

The method is based on \textbf{splitting} the solutionn space into
disjoint sets. It starts by relaxing on the integrality of the variables
and \textbf{solves the LP problem}. If all variables \(x_i\) are
\textbf{integers, the computation ends}. Otherwise one variable
\(x_i \in \mathbb{Z}\) is chosen and its value is assigned to \(k\).
Then the solution space is \textbf{divided into two sets} -- in the
first one we consider \(x_i \leq \lfloor k \rfloor\) and in the second
one \(x_i \geq \lfloor k \rfloor + 1\). The algorithm
\textbf{recursively repeats} computation for the both new sets till
feasible integer solution is found.

As sson as the algorithm finds an integer solutoin, its objective
function value can be used for bounding. The \textbf{node is discarded}
whenever \emph{z}, its (integer or real) objective function value, is
worse than \(z^*\), the value of the best known solution.

\hypertarget{special-cases-of-ilp}{%
\subsection{Special cases of ILP}\label{special-cases-of-ilp}}

Matrix \(\mathbf{A} = [a_{ij}]\) of size \(m/n\) is totally unimodular
if the determinant of every square submatrix of matrix \(\mathbf{A}\) is
equal \(0\), \(+1\) or \(-1\). Matrix \(\mathbf{A}\) is totally
unimodular if each colum contains onne non-zero element or exactly two
non-zero elements \(+1\) and \(-1\).

The ILP problem with a totally unimodular matrix \(\mathbf{A}\) and
integer vector \(b\) can be solved by a simplex algorithm and the
solution will be an integer. Also it can be solved in polynomial time
for this matrix.

\hypertarget{representaion-formulas-as-ilp}{%
\subsection{Representaion formulas as
ILP}\label{representaion-formulas-as-ilp}}

\begin{longtable}[]{@{}ll@{}}
\toprule
\begin{minipage}[b]{0.46\columnwidth}\raggedright
Formula\strut
\end{minipage} & \begin{minipage}[b]{0.48\columnwidth}\raggedright
ILP\strut
\end{minipage}\tabularnewline
\midrule
\endhead
\begin{minipage}[t]{0.46\columnwidth}\raggedright
\(x_1 \Rightarrow \overline{x_3}\)\strut
\end{minipage} & \begin{minipage}[t]{0.48\columnwidth}\raggedright
\(x_1 + x_3 \leq 1\)\strut
\end{minipage}\tabularnewline
\begin{minipage}[t]{0.46\columnwidth}\raggedright
\(x_2 \Rightarrow x_1\)\strut
\end{minipage} & \begin{minipage}[t]{0.48\columnwidth}\raggedright
\(x_2 \leq x_1\)\strut
\end{minipage}\tabularnewline
\begin{minipage}[t]{0.46\columnwidth}\raggedright
\(x_4 \text{ XOR } x_5\)\strut
\end{minipage} & \begin{minipage}[t]{0.48\columnwidth}\raggedright
\(x_4 + x_5 = 1\)\strut
\end{minipage}\tabularnewline
\begin{minipage}[t]{0.46\columnwidth}\raggedright
1 must be chosen but 2 can not\strut
\end{minipage} & \begin{minipage}[t]{0.48\columnwidth}\raggedright
\(x_1 = 1, x_2 = 0\)\strut
\end{minipage}\tabularnewline
\begin{minipage}[t]{0.46\columnwidth}\raggedright
at leat 3 items must be chosen\strut
\end{minipage} & \begin{minipage}[t]{0.48\columnwidth}\raggedright
\(\sum x_i \geq 3\)\strut
\end{minipage}\tabularnewline
\begin{minipage}[t]{0.46\columnwidth}\raggedright
exactly 3 items must be chosen\strut
\end{minipage} & \begin{minipage}[t]{0.48\columnwidth}\raggedright
\(\sum x_i = 3\)\strut
\end{minipage}\tabularnewline
\begin{minipage}[t]{0.46\columnwidth}\raggedright
\((x_1 \text{ AND } x_2) \Rightarrow x_3\)\strut
\end{minipage} & \begin{minipage}[t]{0.48\columnwidth}\raggedright
\(x_1 \cdot x_2 \leq x_3\)\strut
\end{minipage}\tabularnewline
\begin{minipage}[t]{0.46\columnwidth}\raggedright
exactly 2 items can not be chosen\strut
\end{minipage} & \begin{minipage}[t]{0.48\columnwidth}\raggedright
\(\!\begin{aligned}  \sum x_i & \leq 1 + M \cdot y \\  M \cdot (1 - y) + \sum x_i & \geq 3 \\  y \in \{ 0, 1 \}  \end{aligned}\)\strut
\end{minipage}\tabularnewline
\bottomrule
\end{longtable}

\hypertarget{at-least-k-of-n-constraints-must-hold}{%
\subsubsection{\texorpdfstring{At least \(K\) of \(N\) constraints must
hold}{At least K of N constraints must hold}}\label{at-least-k-of-n-constraints-must-hold}}

\begin{align*}
f(x_1, x_2, \dots, x_n) &\leq b_1 + M \cdot y_1 \\
f(x_1, x_2, \dots, x_n) &\leq b_2 + M \cdot y_2 \\
& \vdots \\
f(x_1, x_2, \dots, x_n) &\leq b_N + M \cdot y_N \\
\sum_{i = 1}^N y_i &= N - K
\end{align*}

\end{document}
